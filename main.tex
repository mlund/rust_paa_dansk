\documentclass{article}
\usepackage[a6paper, left=1cm, right=1cm, top=1cm, bottom=1.5cm]{geometry}
\usepackage{xcolor}
\usepackage[colorlinks]{hyperref}
\usepackage{svg}
\usepackage{longtable}
\usepackage{minted}
\usemintedstyle{perldoc}
\newminted{rust}{fontsize=\footnotesize}
\newmintinline{rust}{}
\newminted{bash}{fontsize=\footnotesize}
\renewcommand*\contentsname{Indholdsfortegnelse}
%\usepackage{graphicx} % Required for inserting images

\author{Mikael Lund}
\title{EN HÅNDSRÆKNING\\når du programmerer Rust}
\date{}

\begin{document}

\maketitle
\begin{center}
\includesvg[width=0.9\columnwidth]{fig/Original_Ferris.svg}
\end{center}
\clearpage
\newpage
\tableofcontents

% BOOKLET PRINTING
%https://ftp.acc.umu.se/mirror/CTAN/graphics/pgf/contrib/pgfmorepages/pgfmorepages.pdf

%----------------------------------------

\newpage
\section*{Forord}
\addcontentsline{toc}{section}{Forord}

Den der har haft tysk i folkeskolen vil måske huske et lille orange hæfte med den tillids\-vækkende titel \emph{``En hånds\-rækning når du skriver tysk''}.
Knud Bjerres oversigt over tysk grammatik har i årtier været en redningsplanke for kommende tysk\-skrivere, og det er derfra inspiration til formatet på dette hæfte er hentet.

Rigtigt meget er skrevet om Rust men, ikke overraskende, ganske lidt på dansk.
Hæftet indeholder oversættelser af programmeringsbegreber som 
 mange sikkert vil finde særprægede, da dansk sjældent anvendes i den sammenhæng.
Hvordan oversættes f.eks. ord som \emph{trait} og \emph{linter}?
Jeg syntes det ville være ærgerligt at skrinlægge dansk for emnet, og har forsøgt anvende eksisterende oversættelse af f.eks. C og, når det var hensigtsmæssigt, tillagt nytilkomne ord fra Rust.
Grænserne for \emph{hensigtsmæssig} er naturligvis vide, så på for\-hånd undskyld.

Mål\-gruppen er den lejlig\-heds\-vise Rust-udøver der, ligesom undertegnede, ruster lidt til mellem hver indsats. God fornøjelse!

\vspace{0.5cm}
M.L, Malmø 2023.

%----------------------------------------

\newpage
\section*{Pakke- og byggesystemet \rustinline{cargo}}
\addcontentsline{toc}{section}{Pakkesystemet \rustinline{cargo}}
Opsættes vha. filen \rustinline{Cargo.toml}, og anvendes til en mængde opgaver:

\begin{bashcode}
cargo new {--lib}   # skab et nyt projekt
cargo add <package> # tillæg forudsætning
cargo build         # kompiler kode
cargo run           # kompiler og kør kode
cargo test          # kør test
cargo clippy        # opdag fnuller
cargo fmt           # formater kode
cargo doc           # dokumenter kode
cargo publish       # offentliggør på crates.io

# kør `examples/sprite.rs` i udgivelsestilstand
cargo run --release --example sprite
\end{bashcode}

\noindent Få styr på kompileren\footnote{Rust installeres fra\\
https://www.rust-lang.org/tools/install}:

\begin{bashcode}
rustup update               # opdater
rustup toolchain list       # værktøjsoversigt
rustup default stable       # anvend stabil rust (anbefalet)
rustup default nightly      # anvend natlig rust
rustup override set nightly # køres i projektkatalog
\end{bashcode}

%----------------------------------------
%\newpage
%\section*{Moduler og projektorganisation}
%\addcontentsline{toc}{subsection}{Moduler og projektorganisation}


%----------------------------------------
\newpage
\section*{Strukturer}
\addcontentsline{toc}{section}{Strukturer (\rustinline{struct})}

Om muligt, tilføj (1) typiske \emph{træk} med \rustinline{derive} attributten, og
(2) tillad evaluering under kompilering vha. \rustinline{const} \rustinline{fn}.

\begin{rustcode}
/// Navngivet struktur
#[derive(Debug, Default, Copy, Clone, Eq, PartialEq)]
struct Position { height: f64 };
let pos = Position::default(); // nulstillet højde

/// Enheds-lignende struktur (eng.: unit-like)
struct StayAway;

/// Tuple-lignende struktur
struct Volume(u8);

impl Volume {
    /// Tilknyttet funktion (defineret for en type)
    pub const fn new(volume: u8) -> Self {
        Self {volume}
    }
    /// Tilknyttet metode (kaldet på en type)
    pub fn mute(&mut self) {
        self.0 = 0;
    }
    /// Tilknyttet konstant
    const MAX: Volume = Self::new(u8::MAX);
}
\end{rustcode}

%----------------------------------------
\newpage
\section*{Optællingstyper}
\addcontentsline{toc}{section}{Optællingstyper (\rustinline{enum})}

Modsat C kan \rustinline{enum} have
(1) tilknyttede funktioner,
(2) \emph{varianter} med \emph{data}, samt
(3) være \emph{generiske}, f.eks. \rustinline{enum} \rustinline{Option<T>}.
En enhedsenum\footnote{\rustinline{enum} bestående udelukkende af enhedsvarianter (som i C).} kan omtolkes til unik heltals \emph{diskriminant} vha. \rustinline{as}\footnote{Pakken \rustinline{num_enum} tilbyder avancerede konverteringer.}.

\begin{rustcode}
#[derive(Copy, Clone, Debug, PartialEq)]
#[repr(u8)] // eksplicit diskriminanttype
enum Message {
    Quit = 2, // enhedsvariant; eksplicit diskriminant
    Position {x: i32, y: i32} = 3, // struct-lignende
    Color(u8), // tuple variant; implicit diskriminant
}

let message = Message::Position { x: 200, y: 300 };
assert_eq!(std::mem::discriminant(&Message::Color(0)),
           std::mem::discriminant(&Message::Color(1)));

// Mønstergenkendelse
match message {
    Message::Quit => format!("quit"),
    Message::Position(x, y) => format!("pos {} {}", x, y),
    Message::Color(color) => format!("color {}", color),
}
\end{rustcode}

%----------------------------------------
\newpage
\subsection*{Eksempler fra standardbiblioteket}
\addcontentsline{toc}{subsection}{fra standardbiblioteket}
\rustinline{Option<T>} er en indbygget og meget udbredt optællingstype med to varianter:
\rustinline{None} og \rustinline{Some(T)}.
Her nogle eksempler på hvordan denne anvendes til at håndtere evt. manglende data for \rustinline{number: Option<i32>}:

\begin{itemize}

\item
\begin{rustcode}
if let Some(n) = number {...} else {...}
\end{rustcode}

\begin{rustcode}
match number { // vha. mønstergenkendelse
    None => { /* håndter manglende data */ }
    Some(n) => { /* gør noget med n */ }
}
\end{rustcode}

\item Tidlig afbrydelse:
\begin{itemize}
\item \rustinline{?} operator sender \rustinline{None} til
kaldende funktion:
\begin{rustcode}
fn f(n: i32) -> Option<i32> {
    Some(n.checked_add(3)?.checked_mul(2)?)
}
\end{rustcode}

\item
\begin{rustcode}
let Some(n) = number else { panic!(...); }
// ... gør noget med n
\end{rustcode}
\end{itemize}

\item
\begin{rustcode}
number.map(i32::abs); // -> None eller Some(|n|)
\end{rustcode}

\item
\begin{rustcode}
number.unwrap_or(12); // -> n eller 12 hvis None    
\end{rustcode}

\item
\begin{rustcode}
let number = Some(12);
assert!(matches!(number, Some(n) if n > 2));
\end{rustcode}

\end{itemize}


%----------------------------------------
\newpage
\section*{Træk}
\addcontentsline{toc}{section}{Træk (\rustinline{trait})}
Definer brugergrænseflader vha. \emph{træk}\footnote{Som i \emph{karaktertræk} eller \emph{særtræk}.} eller \rustinline{trait}s:

\begin{rustcode}
trait Playable {
    fn new() -> Self;         // tilknyttet funktion
    fn play(&elf, song: u32); // tilknyttet metode
    fn play_first(&self) {    // foruddefineret metode
        self.play(0);         // (kan overskrives)
    }
}

struct VinylPlayer;
impl Playable for VinylPlayer {
    fn play(&self, song: u32) {}
    ...
}

let player: Box<dyn Playable> = Box::new(VinylPlayer::new());
player.play_first();
\end{rustcode}

%----------------------------------------

\subsubsection*{Trækafgrænsning (eng:. \emph{trait bound})}
\addcontentsline{toc}{subsubsection}{trækafgrænsning}
Begræns træk vha. \emph{trækafgrænsning}

\begin{rustcode}
fn f1(player: &dyn Playable) {}   // dynamisk
fn f2<T: Playable>(player: &T) {} // statisk
fn f3<T>(player: &T)
where
    T: Playable + Default,
{} // statisk; tydeligere ved flere argumenter
\end{rustcode}

%----------------------------------------
\subsubsection*{Trækobjekter (eng.: \emph{trait objects})}
\addcontentsline{toc}{subsubsection}{trækobjekter}

Funktionen \rustinline{fn f1(p: &dyn Playable)} anvender \emph{dynamisk afvikling}\footnote{eng: \emph{dynamic dispatch}} og kan tage imod ethvert objekt hvor \rustinline{Playable} er implementeret. \rustinline{&dyn Playable} er et \emph{trækobjekt} og skabes sådan her:
\begin{rustcode}
let player = VinylPlayer as &dyn Playable;
\end{rustcode}

For at skabe et trækobjekt kræves det at trækket er \emph{objektsikkert} hvilket \underline{blandt andet} forudsætter at\footnote{\url{https://doc.rust-lang.org/reference/items/traits.html\#object-safety}}:
\begin{itemize}
    \item trækket ikke har tilknyttede konstanter, ej heller tilknyttede generiske typer.
    \item trækket ikke har \rustinline{Sized} som supertræk.
    \item at alle tilknyttede funktioner er \emph{afviklingsbare fra et trækobjekt}, \underline{eller} er eksplicit markerede som ikke-afviklingsbare. Sidstnævnte gøres ved at tillægge
    trækafgrænsningen \rustinline{where Self: Sized}
    i definitionen af tilknyttede funktioner.
\end{itemize}

Følgende træk er objektsikkert, men viste funktioner kan \emph{ikke} afvikles fra et trækobjekt:
\begin{rustcode}
trait IkkeAfviklingsbar {
    fn foo() where Self: Sized {}
    fn returns(&self) -> Self where Self: Sized
    fn param(&self, other: Self) where Self: Sized {}
    fn typed<T>(&self, x: T) where Self: Sized {}
}
\end{rustcode}


%----------------------------------------
\newpage
\subsubsection*{Træk fra standard\-biblioteket}
\addcontentsline{toc}{subsubsection}{fra standardbiblioteket}

\rustinline{std} og \rustinline{core} tilbyder en række træk som grænse\-flader\-brugere forventer er implementerede.
Hvis muligt, anvend \rustinline{#[derive(...)]} attributten.

\begin{itemize}
\begin{scriptsize}
\item Sammenligning: \rustinline{Eq}, \rustinline{PartialEq}, \rustinline{Ord}, \rustinline{PartialOrd}.
\item Konvertering: \rustinline{From}, \rustinline{Into}, \rustinline{ToString}.
\item \rustinline{Clone}, skab ny \rustinline{T} fra \rustinline{&T} vha. kopiering.
\item \rustinline{Copy}, \emph{kopier} i stedet for at \emph{flytte}.
\item \rustinline{Default}, definer standardværdi for ny variabel.
\item \rustinline{Debug} til formattering vha. \rustinline|{:?}|.
\item \rustinline{Drop} hvis ressourcer manuelt skal frigives når variablen frigives.
\item \rustinline{Iterator}. Der er enkelt at skrive en iterator, ved bare at definere \rustinline{fn next()}.
\end{scriptsize}
\end{itemize}

%----------------------------------------
\newpage
\subsection*{Ordbog}
\addcontentsline{toc}{section}{Ordbog}

\begin{longtable}{l|l}
\emph{engelsk} & \emph{dansk}\\\hline
associated function & tilknyttet funktion \\
attribute & attribut \\
cast & omtolke \\
compiler & kompiler \\
enum & optællingstype \\
enumeration & enumeration \\
interface & grænseflade \\
lint & fnuller \\
keyword & nøgleord \\
pointer & peger \\
trait & træk \\
trait bound & trækafgrænsning \\
type declaration & typedeklaration (C) \\
unit-like enum & enhedsenum \\
user & bruger \\
\label{tab:ordbog}
\end{longtable}


\end{document}
